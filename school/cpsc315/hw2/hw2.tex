\documentclass[12pt,a4paper]{article}

\usepackage{setspace}
\usepackage{indentfirst}

\setlength{\parindent}{0.5in}
\onehalfspacing

\title{CPSC 315 - Homework \#2}
\author{Chris Nutter\thanks{Dedicated to @QuesoGrande a.k.a. Jared D.}}

% --> Here we go, satellite radio, y'all get hit with a...

\begin{document}

\begin{center}
    {\textbf{\LARGE CPSC 315 - Homework \#2}
        \vskip 6pt
     \large Chris Nutter}
\end{center}

\section{Chapter 2 Summary}
\indent Ethics are one of (if not) the most important concepts in the business field. The chapter talks a lot about different ethical point of views, where they come from, and how they are important in defining morality or what is right and wrong to each person.\\
\indent Ethical relativism is the theory that there is no moral right and wrong and that it is dependent on the person themselves; subjective relativism is that each person has decides what is right and wrong themselves. Cultural relativism is the theory of what is truly right and what is truly wrong rests on society itself and that no individual decides it on their own. Divine command theory is decided around good actions within God's parameters. Of course this more is reliant on religious affiliation (possibly God as the concept of a greater person than thyself). Ethical egoism is focusing on on each person's self-interest rather than other (egotistical complex). Kantianism is the thought that people's actions should be guided by moral laws which are universal laws (similar to the 10 Commandments). Act Utilitarianism is contradictory to Kantianism in that if the benefits exceed the harms it's a good action. Rule Utilitarianism is the theory that we should try to strive to make everyone happy whether that be making sacrifices or not. Social contract theory is the idea that everyone in a civilized society exists if the establishment of rules is in place and a government is capable of enforcing the rules. This concept is very important because it has been adopted by the United States during the creation of the nation. Rawl's theory of justice dictates that competition, if done properly, can help benefit production in social aspects.\\
\indent These various aspects of ethics can be appropriately adapted towards the betterment of not just businesses but life as a whole. We adopt many theories dating back to the Renaissance to this day because of how accurately they can and have been portrayed Understanding ethics and ethical dilemmas allows proper ideas to flourish and business connections to be made.
    \begin{center}\line(1,0){250}\end{center}

\end{document} 
